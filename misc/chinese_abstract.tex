	
\begin{chineseabstract}
视觉问答是给定一张图片和一个图像相关的自然语言问题,输出问题答案的人工智能任务。跨领域的视觉问答接近通用人工智能,有很高的研究价值和广阔的应用场景。按照是否引入外源知识库,现有模型分为联合嵌入模型和基于知识库的模型,这两类模型在视觉问答任务中均有不错的表现。然而主流的联合嵌入模型存在数据集依赖、网络容量小和文本表征能力不足的缺陷。另一方面,通过引入外源知识库,基于知识库的模型克服了联合嵌入模型的网络容量限制,能回答涉及常识或外源知识的推理问题。但其需要通过人工构建知识库查询语句,极大的限制了模型的泛化能力。本文分别改进了联合嵌入模型的文本特征化方法和基于知识库的模型的通用性问题,主要包括以下内容:

1)引入动态词向量改进联合嵌入模型的文本特征化方法。目前的联合嵌入模型的文本特征化方法仍然使用静态词向量方法,考虑到静态词向量无法有效表征一词多义和一词多用的情况,文本在视觉问答模型引入动态词向量,并结合Faster R-CNN和注意力机制,提出了基于动态词向量的联合嵌入模型(N-KBSN)。实验结果证明动态词向量能实现更好的文本特征表示,进而提高准确率。N-KBSN在VQA2.0数据集上的的单模准确率已经超过2019年公开挑战的冠军模型。

2)构建了一个知识库图嵌入模块,以扩展基于知识的模型的通用性。本文构建的知识库图嵌入模块分别从图像和文本中提取核心实体,并映射为知识库实体,再以核心实体为中心提取出子图,并将子图转换为低维向量,实现子图嵌入。为了实现好的子图嵌入,我们首先从DBpedia中提取了两个具有丰富语义的实验知识库:DBV和DBA。并基于这两个知识库,选取了一系列知识库嵌入模型进行链路预测实验。实验结果显示,DBV知识库的实体间具有清晰的对应关系,能实现优异的节点嵌入。并且TransE模型能实现很好的知识库嵌入,因此我们以TransE为核心构建了知识库图嵌入模块。

3)合并知识库图嵌入模块和N-KBSN模型,构建了基于知识库图嵌入的视觉问答模型(KBSN)。在多个数据集上的实验结果证明,知识库图嵌入模块提高了视觉问答的准确率。尤其在面对需要常识或外源知识的复杂问题时,准确率提升明显。

\chinesekeyword{视觉问答,联合嵌入模型,知识库,N-KBSN,KBSN}
\end{chineseabstract}

