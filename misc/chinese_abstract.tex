	
\begin{chineseabstract}
得益于神经网络带来的机器视觉和自然语言处理的快速发展,视觉问答是人工智能领域近几年兴起的热门研究方向。视觉问答是给定一张图片和一个图像相关的自然语言问题,输出问题答案的人工智能任务。虽然由于图像内容的复杂性和问题的开放性使得任务充满挑战,但其相较于人工智能中零散的任务,更接近通用人工智能,因此视觉问答模型的研究具有高度的研究价值和广阔的应用场景。目前已经提出的模型按照是否引入外源知识库可以分为联合嵌入模型和基于知识库的模型两类,本文将分别对这两种模型进行研究。本文首先对视觉问答的国内外研究状况做了详细的调查研究,对其涉及的模型、数据集、知识库均作出了系统性的分析。针对已有的联合嵌入模型使用静态词向量的缺陷,本文使用动态词向量对其进行改进,并结合Faster R-CNN和注意力机制,构建了None KB-Specific Network(N-KBSN)模型。为了进一步提高模型的准确性,文章首次提出引入知识库的图嵌入,在N-KBSN模型的基础上,构建了KB-Specific Network(KBSN)模型。在多个数据集上的实验结果显示,本文改进的动态词向量能够实现提供更好的文本特征,而引入的知识库图嵌入也显著提高了结果的准确率。另外,本文还创建了两个从DBpedia提取的、具有丰富语义的知识库dbv和dba,这两个知识库为以后的知识库图嵌入研究提供数据集支持。

\chinesekeyword{视觉问答,联合嵌入模型,知识库,N-KBSN,KBSN}
\end{chineseabstract}

