
\thesisacknowledgement
在研究生生涯即将结束之际,回看过去三年个人在学术和生活上的成长得失,内心即充满了对各位师友提供的慷慨帮助和精神支持的无限感激,又满怀着对未来的殷切期盼和希望各自安好的美好祝福。因此在论文的结束,希望借此小段略表心中的感激和热切。

首先我要感谢研究生导师郑文锋副教授相识五年以来的一路支持和批判。最初的相识是富有戏剧性的桥段,也必将影响我终身。在课上,我被郑老师对于科学问题和社会问题的独特视角所吸引,在课下,多次的讨论也均引人深思。也正是在多次的交互中,我的思维角度和视野逐渐开阔,并遵循着实证的思路开始思想重塑。研究生阶段的学术探索是在宽松的环境中展开的,长期的学术讨论也帮助我建立起了问题选取、方法定位、实验实施、论文撰写等科学研究的基本方法,本文的研究也离不开老师在选题和实验阶段的帮助,在此特别感谢。

其次,在科学研究思路和内容呈现形式上的提高,我也必须感谢实验室的其他老师,杨波副教授、刘珊副教授和李晓璐博士。每一位老师都从不同的侧面向我传达着作为一个研究者应有的态度和行为模式,这些彰显着他们价值取向的行为也帮助我确立起自我价值。对于一个即将迎接更多科研挑战和生活不确定性的年轻人而言,那些言传身教都难能可贵,尤感敬意。

再者,同实验室其他小伙伴的存在也是研究生生活的一抹亮色,大家长时间的陪伴、定期的聚会、相互的鼓励支持以及每个人独特的人格魅力都是我这三年快乐和幸福的重要来源,也必将成为未来可供追忆的幸福时候。感谢大家这一路的相伴,祝福每一位都能够在自己的人生中安稳而幸福,感谢石天一、张洁勤、肖烨、王爽、王杨、尹超、苗旺、陈阳、徐聪聪。

最后,祝愿答辩组和评阅老师阖家幸福。

